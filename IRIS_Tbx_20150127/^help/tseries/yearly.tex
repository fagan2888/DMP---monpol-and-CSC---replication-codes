

    \filetitle{yearly}{Display tseries object one calendar year per row}{tseries/yearly}

	\paragraph{Syntax}

\begin{verbatim}
yearly(X)
\end{verbatim}

\paragraph{Input arguments}

\begin{itemize}
\itemsep1pt\parskip0pt\parsep0pt
\item
  \texttt{X} {[} tseries {]} - Tseries object that will be displayed one
  full year of observations per row.
\end{itemize}

\paragraph{Description}

The functon \texttt{yearly} currently works for tseries with monthly,
bi-monthly, quarterly, and half-yearly frequency only.

\paragraph{Example}

Create a quarterly tseries, and use \texttt{yearly} to display it one
calendar year per row.

\begin{verbatim}
>> x = tseries(qq(2000,3):qq(2002,2),@rand)
x =
    tseries object: 8-by-1
    2000Q3:  0.95537
    2000Q4:  0.68029
    2001Q1:  0.86056
    2001Q2:  0.93909
    2001Q3:  0.68019
    2001Q4:  0.91742
    2002Q1:  0.25669
    2002Q2:  0.88562
    ''
    user data: empty
>> yearly(x)
    tseries object: 8-by-1
    2000Q1-2000Q4:        NaN           NaN     0.9553698     0.6802907
    2001Q1-2001Q4:  0.8605621     0.9390935      0.680194     0.9174237
    2002Q1-2002Q4:  0.2566917     0.8856181           NaN           NaN
    ''
    user data: empty
\end{verbatim}


