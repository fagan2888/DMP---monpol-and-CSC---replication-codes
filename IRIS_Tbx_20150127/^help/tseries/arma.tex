

    \filetitle{arma}{Apply ARMA model to input series}{tseries/arma}

	\paragraph{Syntax}

\begin{verbatim}
X = arma(E,Ar,Ma)
X = arma(E,Ar,Ma,Range)
\end{verbatim}

\paragraph{Input arguments}

\begin{itemize}
\item
  \texttt{E} {[} tseries {]} - Input time series that will be run
  through an ARMA model.
\item
  \texttt{Ar} {[} numeric \textbar{} empty {]} - Row vector of AR
  coefficients; if empty, \texttt{Ar = 1}; see Description.
\item
  \texttt{Ma} {[} numeric \textbar{} empty {]} - Row vector of MA
  coefficients; if empty, \texttt{Ma = 1}; see Description.
\item
  \texttt{Range} {[} numeric \textbar{} \texttt{@auto} {]} - Range on
  which the input series will be constructed; if not specified or
  \texttt{@auto}, the range will be determined based on the input time
  series, \texttt{E}.
\end{itemize}

\paragraph{Output arguments}

\begin{itemize}
\itemsep1pt\parskip0pt\parsep0pt
\item
  \texttt{X} {[} tseries {]} - Output time series constructed by running
  an ARMA model through the input series.
\end{itemize}

\paragraph{Options}

\paragraph{Description}

The output series is constructed as follows:

\[ A(L) X_t = M(L) E_t \]

where $A(L)$ and $M(L)$ are polynomials in lag operator defined by the
vectors \texttt{Ar} and \texttt{Ma}. In other words,

\begin{verbatim}
X(t) = ( -Ar(2)*X(t-1) - Ar(3)*X(t-2) - ...
       + Ma(1)*E(t) + Ma(2)*E(t-1) + ... ) / Ar(1);
\end{verbatim}

\paragraph{Example}

Construct an AR(1) process with autoregression coefficient 0.8, based on
normally distributed innovations.

\begin{verbatim}
E = tseries(1:20,@randn);
X = arma(E,[1,-0.8],[]);
\end{verbatim}


