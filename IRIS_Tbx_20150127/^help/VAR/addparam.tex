

    \filetitle{addparam}{Add VAR parameters to a database (struct)}{VAR/addparam}

	\paragraph{Syntax}

\begin{verbatim}
D = addparam(V,D)
\end{verbatim}

\paragraph{Input arguments}

\begin{itemize}
\item
  \texttt{V} {[} VAR {]} - VAR object whose parameter matrices will be
  added to database (struct) \texttt{D}.
\item
  \texttt{D} {[} struct {]} - Database to which the model parameters
  will be added.
\end{itemize}

\paragraph{Output arguments}

\begin{itemize}
\itemsep1pt\parskip0pt\parsep0pt
\item
  `D {[} struct {]} - Database with the VAR parameter matrices added.
\end{itemize}

\paragraph{Description}

The newly created database entries are named \texttt{A\_} (transition
matrix), \texttt{K\_} (constant terms), \texttt{J\_} (coefficient matrix
in front of exogenous inputs), \texttt{B\_} (matrix of instantaneous
whock effects), and \texttt{Cov\_} (covariance matrix of shocks). Be
aware that all existing database entries in \texttt{D} named
\texttt{A\_}, \texttt{K\_}, \texttt{B\_}, or \texttt{Omg\_} will be
overwritten.

\paragraph{Example}

\begin{verbatim}
D = struct();
D = addparam(V,D);
\end{verbatim}


