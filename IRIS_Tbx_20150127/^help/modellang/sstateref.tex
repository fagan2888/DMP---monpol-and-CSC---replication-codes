

    \filetitle{\&}{Reference to the steady-state level of a variable}{modellang/sstateref}

	\paragraph{Syntax}

\begin{verbatim}
&VariableName
$VariableName
&VariableName{K}
$VariableName{K}
\end{verbatim}

\paragraph{Description}

Use either a \texttt{\&} or \texttt{\$} sign in front of a variable name
to create a reference to that variable's steady-state level in
transition or measurement equations. The two signs, \texttt{\&} and
\texttt{\$}, are interchangeable. Steady-state references may only be
used in nonlinear models.

The steady-state reference can include a time shift (a lag or a lead),
\texttt{K}. In that case, the steady-state value will be adjusted for
steady-state growth backward or forward accordingly.

The steady-state reference will be replaced

\begin{itemize}
\item
  with the variable itself at the time the model's steady state is being
  calculated, i.e.~when calling the function
  \href{model/sstate}{\texttt{sstate}};
\item
  with the actually assigned steady-state value at the time the model is
  being solved, i.e.~when calling the function
  \href{model/solve}{`solve'}'.
\end{itemize}

\paragraph{Example}

\begin{verbatim}
x = rho*x{-1} + (1-rho)*&x + epsilon_x !! x = 1;
\end{verbatim}


