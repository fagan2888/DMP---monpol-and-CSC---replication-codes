

    \filetitle{!links}{Define dynamic links}{modellang/links}

	\paragraph{Syntax}

\begin{verbatim}
!links
   ParameterName := Expression;
   VariableName := Expression;
\end{verbatim}

\paragraph{Syntax with equation
labels}

\begin{verbatim}
!links
   'Equation label' ParameterName := Expression;
   'Equation label' VariableName := Expression;
\end{verbatim}

\paragraph{Description}

The dynamic links relate a particular parameter (or steady-state value)
on the LHS to a function of other parameters or steady-state values on
the RHS. \texttt{Expression} can be any expression involving parameter
names, variables names, Matlab functions and constants, or your own
m-file functions on the path; it must not refer to any lags or leads.
\texttt{Expression} must evaluate to a single number. It is the user's
responsibility to properly handle the imaginary (i.e.~growth) part of
the steady-state values.

The links are automatically refreshed in
\href{model/solve}{\texttt{solve}},
\href{model/sstate}{\texttt{sstate}}, and
\href{model/chksstate}{\texttt{chksstate}} functions, and also in each
iteration within the \href{model/estimate}{\texttt{estimate}} function.
They can also be refreshed manually by calling
\href{model/refresh}{\texttt{refresh}}.

The links must not involve parameters occuring in
\href{modellang/dtrends}{\texttt{!dtrends}} equations that will be
estimated using the \texttt{'outoflik='} option of the
\href{model/estimate}{\texttt{estimate}} function.

\paragraph{Example}

\begin{verbatim}
!links
   R := 1/beta;
   alphak := 1 - alphan - alpham;
\end{verbatim}


