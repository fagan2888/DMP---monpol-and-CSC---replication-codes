

    \filetitle{resample}{Resample from the model implied distribution}{model/resample}

	\paragraph{Syntax}

\begin{verbatim}
Outp = resample(M,Inp,Range,NDraw,...)
Oupt = resample(M,Inp,Range,NDraw,J,...)
\end{verbatim}

\paragraph{Input arguments}

\begin{itemize}
\item
  \texttt{M} {[} model {]} - Solved model object.
\item
  \texttt{Inp} {[} struct \textbar{} empty {]} - Input data (if needed)
  for the distributions of initial condition and/or empirical shocks; if
  not bootstrap is invovled
\item
  \texttt{Range} {[} numeric {]} - Resampling date range.
\item
  \texttt{NDraw} {[} numeric {]} - Number of draws.
\item
  \texttt{J} {[} struct \textbar{} empty {]} - Database with
  user-supplied (time-varying) tunes on std devs, corr coeffs, and/or
  means of shocks.
\end{itemize}

\paragraph{Output arguments}

\begin{itemize}
\itemsep1pt\parskip0pt\parsep0pt
\item
  \texttt{Outp} {[} struct {]} - Output database with resampled data.
\end{itemize}

\paragraph{Options}

\begin{itemize}
\item
  \texttt{'bootstrapMethod='} {[} \emph{\texttt{'efron'}} \textbar{}
  \texttt{'wild'} \textbar{} numeric {]} - Numeric options correspond to
  block sampling methods. Use a positive integer to specify a fixed
  block length, or a value strictly between zero and one to specify
  random block lengths based on a geometric distribution.
\item
  \texttt{'deviation='} {[} \texttt{true} \textbar{}
  \emph{\texttt{false}} {]} - Treat input and output data as deviations
  from balanced-growth path.
\item
  \texttt{'dtrends='} {[} \emph{\texttt{@auto}} \textbar{} \texttt{true}
  \textbar{} \texttt{false} {]} - Add deterministic trends to
  measurement variables.
\item
  \texttt{'method='} {[} \texttt{'bootstrap'} \textbar{}
  \emph{\texttt{'montecarlo'}} {]} - Method of randomising shocks and
  initial condition.
\item
  \texttt{'progress='} {[} \texttt{true} \textbar{}
  \emph{\texttt{false}} {]} - Display progress bar in the command
  window.
\item
  \texttt{'randomInitCond='} {[} \emph{\texttt{true}} \textbar{}
  \texttt{false} \textbar{} numeric {]} - Randomise initial condition; a
  number means the initial condition will be simulated using the
  specified number of extra pre-sample periods.
\item
  \texttt{'stateVector='} {[} \emph{\texttt{'alpha'}} \textbar{}
  \texttt{'x'} {]} - When resampling initial condition, use the
  transformed state vector, \texttt{alpha}, or the vector of original
  variables, \texttt{x}; this option is meant to guarantee replicability
  of results.
\item
  \texttt{'svdOnly='} {[} \texttt{true} \textbar{} \emph{\texttt{false}}
  {]} - Do not attempt Cholesky and only use SVD to factorize the
  covariance matrix when resampling initial condition; only applies when
  \texttt{'randomInitCond=' true}.
\end{itemize}

\paragraph{Description}

When you use wild bootstrap for resampling the initial condition, the
results are based on an assumption that the mean of the initial
condition is the asymptotic mean implied by the model (i.e.~the steady
state).

\paragraph{References}

\begin{enumerate}
\def\labelenumi{\arabic{enumi}.}
\itemsep1pt\parskip0pt\parsep0pt
\item
  Politis, D. N., \& Romano, J. P. (1994). The stationary bootstrap.
  Journal of the American Statistical Association, 89(428), 1303-1313.
\end{enumerate}

\paragraph{Example}


