

    \filetitle{bn}{Beveridge-Nelson trends}{model/bn}

	\paragraph{Syntax}

\begin{verbatim}
Outp = bn(M,Inp,Range,...)
\end{verbatim}

\paragraph{Input arguments}

\begin{itemize}
\item
  \texttt{M} {[} model {]} - Solved model object.
\item
  \texttt{Inp} {[} struct \textbar{} cell {]} - Input data on which the
  BN trends will be computed.
\item
  \texttt{Range} {[} numeric {]} - Date range on which the BN trends
  will be computed.
\end{itemize}

\paragraph{Output arguments}

\begin{itemize}
\itemsep1pt\parskip0pt\parsep0pt
\item
  \texttt{Outp} {[} struct \textbar{} cell {]} - Output data with the BN
  trends.
\end{itemize}

\paragraph{Options}

\begin{itemize}
\item
  \texttt{'deviations='} {[} \texttt{true} \textbar{}
  \emph{\texttt{false}} {]} - Input and output data are deviations from
  balanced-growth paths.
\item
  \texttt{'dtrends='} {[} \emph{\texttt{@auto}} \textbar{} \texttt{true}
  \textbar{} \texttt{false} {]} - Measurement variables in input and
  output data include deterministic trends specified in
  \href{modellang/dtrends}{\texttt{!dtrends}} equations.
\end{itemize}

\paragraph{Description}

The BN decomposition is accurate only if the input data have been
generated using unanticipated shocks.

\paragraph{Example}


