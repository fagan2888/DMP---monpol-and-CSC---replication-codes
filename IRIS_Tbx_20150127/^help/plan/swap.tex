

    \filetitle{swap}{Swap endogeneity and exogeneity of variables and shocks}{plan/swap}

	\paragraph{Syntax}

\begin{verbatim}
P = swap(P,ExogList,EndogList,Dates)
P = swap(P,ExogList,EndogList,Dates,Sigma)
\end{verbatim}

\paragraph{Input arguments}

\begin{itemize}
\item
  \texttt{P} {[} plan {]} - Simulation plan.
\item
  \texttt{ExogList} {[} cellstr \textbar{} char {]} - List of variables
  that will be exogenized.
\item
  \texttt{EndogList} {[} cellstr \textbar{} char {]} - List of shocks
  that will be endogenized.
\item
  \texttt{Dates} {[} numeric {]} - Dates at which the variables and
  shocks will be exogenized/endogenized.
\item
  \texttt{Sigma} {[} numeric {]} - Anticipation mode (real or imaginary)
  for the endogenized shocks, and their numerical weight (used in
  underdetermined simulation plans); if omitted, \texttt{Sigma = 1}.
\end{itemize}

\paragraph{Output arguments}

\begin{itemize}
\itemsep1pt\parskip0pt\parsep0pt
\item
  \texttt{P} {[} plan {]} - Simulation plan with new information on
  exogenized variables and endogenized shocks included.
\end{itemize}

\paragraph{Description}

The function \texttt{swap} is equivalent to the following separate calls
to functions \texttt{exogenize} and \texttt{endogenize}:

\begin{verbatim}
p = exogenize(p,ExogList,Dates);
p = endogenize(p,EndogList,Dates);
\end{verbatim}

or

\begin{verbatim}
p = exogenize(p,ExogList,Dates);
p = endogenize(p,EndogList,Dates,Sigma);
\end{verbatim}

if the input argument \texttt{Sigma} is provided.

\paragraph{Example}


