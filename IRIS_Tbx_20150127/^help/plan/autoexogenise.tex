

    \filetitle{autoexogenise}{Exogenise variables and automatically endogenise corresponding shocks}{plan/autoexogenise}

	\paragraph{Syntax}

\begin{verbatim}
P = autoexogenise(P,List,Dates)
P = autoexogenise(P,List,Dates,Sigma)
\end{verbatim}

\paragraph{Input arguments}

\begin{itemize}
\item
  \texttt{P} {[} plan {]} - Simulation plan.
\item
  \texttt{List} {[} cellstr \textbar{} char \textbar{} \texttt{@all} {]}
  - List of variables that will be exogenised; these variables must have
  their corresponding shocks assigned, see
  \href{modellang/autoexogenise}{\texttt{!autoexogenise}}; \texttt{@all}
  means all autoexogenised variables defined in the model object will be
  exogenised.
\item
  \texttt{Dates} {[} numeric {]} - Dates at which the variables will be
  exogenised.
\item
  \texttt{Sigma} {[} \texttt{1} \textbar{} \texttt{1i} \textbar{}
  numeric {]} - Anticipation mode (real or imaginary) for endogenized
  shocks, and their numerical weight (used in underdetermined simulation
  plans); if omitted, \texttt{Sigma = 1}.
\end{itemize}

\paragraph{Output arguments}

\begin{itemize}
\itemsep1pt\parskip0pt\parsep0pt
\item
  \texttt{P} {[} plan {]} - Simulation plan with new information on
  exogenised variables and endogenised shocks included.
\end{itemize}

\paragraph{Description}

\paragraph{Example}


